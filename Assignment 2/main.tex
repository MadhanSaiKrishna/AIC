\documentclass[11pt]{article}
\usepackage[utf8]{inputenc}
\usepackage[margin=1in]{geometry}
\usepackage{amsmath, amssymb}
\usepackage{graphicx}
\usepackage{float}
\usepackage{booktabs}
\usepackage{hyperref}
\usepackage{bookmark}
\usepackage{caption}

% --- Document Information ---
\title{Analog IC Design: Assignment-2 \\ \large Spring 2026, IIIT Hyderabad}
\author{Name: \underline{\hspace{5cm}} \quad Roll No: \underline{\hspace{3cm}}}
\date{Due Date: February 23, 2026 (18:00 hrs)}

\begin{document}

\maketitle

\section*{Instructions}
\begin{itemize}
    \item Technology: 180 nm SCL technology file.
    \item Content: Schematics, netlists, annotated waveforms, and inference/discussion are required for each solution.
\end{itemize}

\noindent\rule{\textwidth}{0.4pt}

% --- Question 1 ---
\section*{Question 1: n-channel MOSFET Characteristics}
Considering $V_{DS} = 1\text{V}$ and $\frac{W}{L} = \frac{500\text{n}}{180\text{n}}$:

\subsection*{(a) $I_D$ vs $V_{GS}$ and $g_m$ vs $V_{GS}$ Plots}
\textbf{Cases:} (i) $V_{SB} = 0\text{V}$, (ii) $V_{SB} = 0.9\text{V}$, (iii) $V_{SB} = -0.9\text{V}$.

\subsubsection*{Schematic and Simulation Setup}
% Insert Schematic description/image here
[Insert Schematic Here]

\subsubsection*{Simulation Results (Plots)}
% Insert Plots here
[Insert super-imposed $I_D$ vs $V_{GS}$ plot here]
[Insert super-imposed $g_m$ vs $V_{GS}$ plot here]

\subsubsection*{Discussion}
% Briefly discuss variations, causes, and analytical match.
\textit{Discussion:} 

\subsection*{(b) Reference Current $I_{D0}$}
For $V_{SB} = 0\text{V}$ and $V_{GS} = 0.9\text{V}$:
\begin{itemize}
    \item $I_{D0} = $ \underline{\hspace{3cm}}
\end{itemize}

\subsection*{(c) $g_m$ Variation for $V_{GS} = 0.9\text{V}$}
% Report variation considering VSB=0 as reference
\begin{itemize}
    \item Case $V_{SB} = 0.9\text{V}$: Variation = 
    \item Case $V_{SB} = -0.9\text{V}$: Variation = 
\end{itemize}

\subsection*{(d) $g_m$ Variation for $I_D = I_{D0}$}
% Report variation considering VSB=0 as reference
\begin{itemize}
    \item Case $V_{SB} = 0.9\text{V}$: Variation = 
    \item Case $V_{SB} = -0.9\text{V}$: Variation = 
\end{itemize}

\newpage

% --- Question 2 ---
\section*{Question 2: Transconductance Analysis}
Using the circuit in Figure 1 with $V_{DD} = 1.8\text{V}$.

\subsection*{(a) $g_m$ vs $I_D$ Curve}
Sweep $I_D$ from $10\,\mu\text{A}$ to $1\text{mA}$ for:
1. $(W/L) = \frac{3\,\mu\text{m}}{1\,\mu\text{m}}$
2. $(W/L) = \frac{0.45\,\mu\text{m}}{0.18\,\mu\text{m}}$

[Insert super-imposed $g_m$ vs $I_D$ plot here]

\textbf{Inference:} Which case is closer to the expected form $g_m = \frac{2I_D}{V_{GS}-V_T}$ and why?

\subsection*{(b) Max Transconductance and Saturation}
\begin{itemize}
    \item $g_{m,max}$ observed: \underline{\hspace{2cm}}
    \item $I_D$ at which $g_m$ saturates: \underline{\hspace{2cm}}
    \item $I_D$ for $g_{m,use} = 0.8 \times g_{m,max}$: \underline{\hspace{2cm}}
\end{itemize}

\subsection*{(c) Scaling Parameters}
To obtain $g_{m,req} = 4 \times g_{m,use}$ while ensuring the same speed:
% Show scaling logic here

\newpage

% --- Question 3 ---
\section*{Question 3: CS Amplifier Design ($g_m/I_D$ Approach)}
\textbf{Specifications:} $A_v > 60$, $f_u = 100\text{MHz}$, $C_L = 1\text{pF}$, $V_{DD} = 1.8\text{V}$, $V^* = \frac{2I_D}{g_m} = 200\text{mV}$.

\subsection*{(a) Channel Length Selection}
[Insert plot of $g_m r_o$ vs $V_{DS}$ for different $L$]
\textbf{Chosen $L$:} \underline{\hspace{2cm}}
\textbf{Output Swing Range:} \underline{\hspace{3cm}}

\subsection*{(b) Process and Temperature Variations}
[Insert $g_m r_o$ vs $V_{DS}$ plot for corners: tt, ff, ss and temps: -40, 25, 80, 125 $^\circ$C]

\subsection*{(c) Design Calculations}
\begin{itemize}
    \item Calculated $g_m$:
    \item Calculated $I_D$:
    \item Calculated $W$:
\end{itemize}

\subsection*{(d) Transient Analysis and THD}
[Insert annotated transient plots]
\textbf{THD for maximum input swing:} \underline{\hspace{2cm}}

\subsection*{(e) AC Analysis}
[Insert AC Response plots]
\begin{itemize}
    \item Verified $f_u$:
    \item $f_{-3dB}$:
    \item AC Gain:
\end{itemize}

\newpage

% --- Question 4 ---
\section*{Question 4: Common Source Amplifier Variants}
$R_L = 10\text{k}\Omega$, $V_{DD} = 1.8\text{V}$, $A_v > 5$, Overdrive $= 200\text{mV}$, $f_{min} = 100\text{Hz}$.

\subsection*{(a) Design Procedure (Resistive Load)}
% Show calculations for M1, M2, I_REF, C_b, R_b
\textbf{Total Power Consumption:} \underline{\hspace{2cm}}

\subsection*{(b) Transient Simulation}
[Insert transient plots for $v_{in} = 10\sin(2\pi \cdot 1000t)\text{mV}$]

\subsection*{(c) AC Response}
[Insert AC plots $1\text{Hz}$ to $1\text{GHz}$]
\textbf{Unity Bandwidth $f_u$:} \underline{\hspace{2cm}}

\subsection*{(d) PMOS Current Source Load}
[Insert design procedure, Transient, and AC plots for $A_v > 15$]

\subsection*{(e) Cascode Amplifier Design}
[Insert design procedure, Transient, and AC plots]
\textbf{Observed Gain:} \underline{\hspace{2cm}}
\textbf{Comparison with Hand Calculations:}

\subsection*{(f) Comparison Table}
\begin{table}[H]
\centering
\caption{Comparison of Amplifier Topologies}
\begin{tabular}{@{}lcccc@{}}
\toprule
\textbf{Topology} & \textbf{Gain} & \textbf{Power} & \textbf{Output Swing} & \textbf{Remarks} \\ \midrule
Resistive Load    &               &                &                       &                  \\
PMOS Load         &               &                &                       &                  \\
Cascode           &               &                &                       &                  \\ \bottomrule
\end{tabular}
\end{table}

\end{document}