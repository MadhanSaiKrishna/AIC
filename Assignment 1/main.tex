\documentclass[12pt,a4paper]{article}
\usepackage[utf8]{inputenc}
\usepackage{amsmath, amssymb}
\usepackage{enumitem}
\usepackage{geometry}
\usepackage{graphicx}
\usepackage{float} 
\geometry{margin=1in}

\title{\textbf{Analog IC Design} \\ \Large Assignment 1}
\author{Chamarthy Madhan Sai Krishna \\ 2023102030}
\date{\today}
\begin{document}
\maketitle

\section*{Question 2}
Use the given SCL model file and plot $I_D$ vs $V_{GS}$ for a $10\,\mu\text{m}/1\,\mu\text{m}$ NMOS transistor and estimate its $V_T$ from the graph for the following cases  
(Suggested reading: \textit{Operation and Modeling of the MOS Transistor}, second ed., by Tsividis, Sections 10.4 and 4.10):

\begin{enumerate}[label=(\alph*)]
    \item $V_{DS} = 50~\text{mV}$ and $V_{GS}$ is swept from $0$ to $1.8~\text{V}$ in steps of $0.1~\text{V}$.
    \item $V_{DS} = 1.8~\text{V}$ and $V_{GS}$ is swept from $0$ to $1.8~\text{V}$ in steps of $0.1~\text{V}$.
    \item Do you observe any difference in $V_T$ values in cases (a) and (b)? If yes, explain why and indicate which one you will use for hand calculations with simple MOS models.
    \item Estimate the technology parameter $\mu C_{\text{ox}}$ from simulations and simple MOS models.
    \item Plot $I_D$ vs $V_{GS}$ for $V_{DS} = 50~\text{mV}$ and $V_{DS} = 1.8~\text{V}$. Clearly mark the subthreshold region on the graphs.  
    From the plots, and using the simple exponential relationship
    \[
        I_D = I_0 \exp\left(\frac{V_{GS}}{\eta V_{\text{thermal}}}\right),
    \]
    estimate the process parameter $\eta$.
\end{enumerate}
\section*{Solution:}
\subsection*{(a) Linear Region ($V_{DS} = 50\text{mV}$):} 
Since $V_{DS}$ is very small ($V_{DS} << V_{GS} - V_{Th}$), the MOSFET is in the triode region. The current is approximately (neglecting the $\alpha$ parameter that's described in Tsividis) modeled as:
$$I_D = \mu_n C_{ox} \frac{W}{L} (V_{GS} - V_T)V_{DS}$$
The transconductance $g_m = \frac{\partial I_D}{\partial V_{GS}}$ is maximum at the point of strongest inversion. A tangent line is drawn to the $I_D$ curve at this maximum slope point. The x-axis intercept of this tangent ($V_{GS,int}$) is identified.
$$V_T = V_{GS,int}$$

\begin{figure}[H]
    \centering
    \includegraphics[width=1\textwidth]{images/2_a_1.png}
    \caption{Plot of $I_D$ vs. $V_{GS}$ for $V_{DS} = 50mV$}
\end{figure}

\begin{figure}[H]
    \centering
    \includegraphics[width=1\textwidth]{images/2_a_2.png}
    \caption{Plot of  $\frac{dI_D}{dV_{GS}}$ vs. $V_{GS}$ for $V_{DS} = 50mV$}
\end{figure}

\begin{figure}[H]
    \centering
    \includegraphics[width=1\textwidth]{images/2_a_3.png}
    \caption{Tangent Line at Maximum Slope Point for $V_{DS} = 50mV$}
\end{figure}

Slope ($m$): $99.697572 \times 10^{-6} \text{ S}$ \\
Point ($V_{GS0}, I_{D0}$): $(1.35 \text{ V}, 78.167927 \text{ }\mu\text{A})$

The equation of the line is given by:
$$y - y_0 = m(x - x_0)$$
Substituting your values (using $\mu\text{A}$ for current):
$$I_D - 78.167927 = 99.697572(V_{GS} - 1.35)$$

$$V_{GS,int} = V_{Threshold} = \frac{56.423795}{99.697572}$$
$$V_{Threshold,linear} \approx \textbf{0.56595} \text{ V}$$
\subsection*{(b) Saturation Region ($V_{DS} = 1.8\text{V}$):}
\begin{figure}[H]
    \centering
    \includegraphics[width=1\textwidth]{images/2_b_1.png}
    \caption{Plot of $I_D$ vs. $V_{GS}$ for $V_{DS} = 1.8V$}
\end{figure}

\begin{figure}[H]
    \centering
    \includegraphics[width=1\textwidth]{images/2_b_2.png}
    \caption{Plot of  $\frac{d\sqrt{I_D}}{dV_{GS}}$ for $V_{DS} = 1.8V$}
\end{figure}

\begin{figure}[H]
    \centering
    \includegraphics[width=1\textwidth]{images/2_b_3.png}
    \caption{Tangent Line at Maximum Slope Point for $V_{DS} = 1.8V$}
\end{figure}

The standard square-law model for an NMOS in saturation ($V_{DS} \ge V_{GS} - V_T$) is:
$$I_D = \frac{1}{2} \mu_n C_{ox} \frac{W}{L} (V_{GS} - V_T)^2$$

To extract $V_T$ through linear extrapolation, we take the square root of both sides of the saturation equation:
$$\sqrt{I_D} = \sqrt{\frac{1}{2} \mu_n C_{ox} \frac{W}{L}} \cdot (V_{GS} - V_T)$$
This matches the equation of a straight line, $y = mx + c$, where: \\
$y$ = $\sqrt{I_D}$ \\
$x$ = $V_{GS}$ \\
Slope ($m$) = $\sqrt{\frac{1}{2} \mu_n C_{ox} \frac{W}{L}}$ \\
Constant ($c$) = $-m \cdot V_T$

Using the slope ($m$) and a point $(V_{GS0}, \sqrt{I_{D0}})$ from the linear region, we can find the intercept.
The linear equation is: $\sqrt{I_D} = m \cdot V_{GS} + c$
The x-intercept occurs when $\sqrt{I_D} = 0$:
$$0 = m \cdot V_{GS,int} + c \implies V_{GS,int} = -\frac{c}{m}$$
Thus, $V_{T,sat} = V_{GS,int}$.

Slope ($m$): $28.472661 \sqrt{\mu\text{A}}/\text{V}$ \\
Point ($V_{GS0}, \sqrt{I_{D0}}$): $(0.54095185\text{ V}, \sqrt{19.296581} \sqrt{\mu\text{A}})$ \\
$\sqrt{I_{D0}} \approx 4.392787 \sqrt{\mu\text{A}}$

Substituting these values into the point-slope form $y - y_0 = m(x - x_0)$ where $y = \sqrt{I_D}$ (in $\mu\text{A}$):
$$\sqrt{I_D} - 4.392787 = 28.472661(V_{GS} - 0.540952)$$
Expanding to find the intercept form:
$$\sqrt{I_D} = 28.472661 \cdot V_{GS} - (28.472661 \times 0.540952) + 4.392787$$
$$\sqrt{I_D} = 28.472661 \cdot V_{GS} - 15.402338 + 4.392787$$
$$\sqrt{I_D} = 28.472661 \cdot V_{GS} - 11.009551$$

$$V_{GS,int} = V_{Threshold,Saturation} \approx \mathbf{0.38667 \text{ V}}$$

\subsection*{(c) Comparison of $V_T$ Values}

Case (a) Linear Region ($V_{DS} = 50\text{mV}$): $V_{T,lin} \approx 0.541\text{V}$ \\
Case (b) Saturation Region ($V_{DS} = 1.8\text{V}$): $V_{T,sat} \approx 0.387\text{V}$

The threshold voltage decreased by approximately $154\text{mV}$ as the drain-source voltage increased from $50\text{mV}$ to $1.8\text{V}$.

This reduction is primarily due to Drain-Induced Barrier Lowering (DIBL). As $V_{DS}$ increases, the high drain potential lowers the potential barrier at the source end of the channel found in these sub-micron devices. Consequently, a lower gate voltage is required to invert the channel, resulting in a lower effective threshold voltage in saturation.

For hand calculations using simple MOS models, I will use the threshold voltage extracted from Case (a) ($V_{DS} = 50\text{mV}$), i.e., $V_T \approx 0.541\text{V}$.
The simple square-law models assume $V_T$ is a constant parameter and do not account for DIBL. Therefore, the value extracted at low $V_{DS}$ represents the intrinsic threshold voltage of the process and is the appropriate choice for basic hand calculations.

\subsection*{(d) Estimation of Technology Parameter $\mu_n C_{ox}$}

To estimate the technology parameter $\mu_n C_{ox}$ (also known as the process transconductance parameter $k'_n$), we use the slopes extracted from the simulation data and relate them to the simple MOS models for both the linear and saturation regions.

\textbf{1. Estimation from the Linear Region ($V_{DS} = 50\text{mV}$) }

In the linear region, the simple MOS model for drain current is:
$$I_D = \mu_n C_{ox} \frac{W}{L} \left[ (V_{GS} - V_T)V_{DS} - \frac{V_{DS}^2}{2} \right]$$
The transconductance ($g_m$), which is the slope of the $I_D$ vs. $V_{GS}$ curve, is:
$$g_{m,lin} = \frac{\partial I_D}{\partial V_{GS}} = \mu_n C_{ox} \frac{W}{L} V_{DS}$$

Using the simulated values:
Slope ($g_{m,lin}$): $99.697572 \times 10^{-6} \text{ A/V}$ \\
Drain-Source Voltage ($V_{DS}$): $0.05 \text{ V}$ \\
Aspect Ratio ($W/L$): $10\mu\text{m} / 1\mu\text{m} = 10$

$$\mu_n C_{ox} = \frac{g_{m,lin}}{(W/L) \cdot V_{DS}} = \frac{99.697572 \times 10^{-6}}{10 \times 0.05}$$
$$\mu_n C_{ox} \approx \mathbf{199.40 \text{ }\mu\text{A/V}^2}$$

\textbf{2. Estimation from the Saturation Region ($V_{DS} = 1.8\text{V}$)}

In the saturation region, the square-law relationship is:
$$I_D = \frac{1}{2} \mu_n C_{ox} \frac{W}{L} (V_{GS} - V_T)^2$$
Linearizing this by taking the square root gives:
$$\sqrt{I_D} = \sqrt{\frac{1}{2} \mu_n C_{ox} \frac{W}{L}} \cdot (V_{GS} - V_T)$$
The slope of the $\sqrt{I_D}$ vs. $V_{GS}$ curve is:
$$m_{sat} = \sqrt{\frac{1}{2} \mu_n C_{ox} \frac{W}{L}}$$

Using the simulated values:
Slope ($m_{sat}$): $28.472661 \text{ mA}^{1/2}/\text{V} = 0.028472661 \text{ A}^{1/2}/\text{V}$ \\
Aspect Ratio ($W/L$): $10$

$$\mu_n C_{ox} = \frac{2 \cdot m_{sat}^2}{W/L} = \frac{2 \cdot (0.028472661)^2}{10}$$
$$\mu_n C_{ox} \approx \frac{2 \cdot 0.00081069}{10} \approx 0.00016213$$
$$\mu_n C_{ox} \approx \mathbf{162.13 \text{ }\mu\text{A/V}^2}$$

\textbf{3. Explanation}

The $\mu_n C_{ox}$ calculated from the linear region ($199.4\text{ }\mu\text{A/V}^2$) is significantly higher than that from the saturation region ($162.1\text{ }\mu\text{A/V}^2$).


\textbf{Mobility Degradation:} In saturation, the vertical electric field (from the gate) and the lateral electric field (from the drain) are much higher. This causes carriers to collide more frequently with the oxide interface, reducing the effective mobility ($\mu_{eff}$) compared to the low-field linear case. \\
\textbf{Velocity Saturation:} At $V_{DS} = 1.8\text{V}$, the carriers reach their scattering-limited velocity, which prevents the current from increasing quadratically as the simple model predicts, leading to a lower "apparent" $\mu_n C_{ox}$ in saturation.
\subsection*{(e) Estimation of subthreshold process parameter $\eta$}

\begin{figure}[H]
    \centering
    \includegraphics[width=1\textwidth]{images/2_e_1.png}
    \caption{Plot of  $\ln(I_D)$ vs. $V_{GS}$ for $V_{DS} = 50mV$}
\end{figure}

Point 1: $(V_{GS1}, \ln(I_{D1})) = (200.867 \text{ mV}, -20.032)$ \\
Point 2: $(V_{GS2}, \ln(I_{D2})) = (270.098 \text{ mV}, -18.026)$

The subthreshold current is given by:
$$I_D = I_0 \exp\left( \frac{V_{GS}}{\eta V_{thermal}} \right)$$
Taking the natural log:
$$\ln(I_D) = \ln(I_0) + \frac{1}{\eta V_{thermal}} \cdot V_{GS}$$
This takes the form of a line $y = mx + c$, where the slope $m$ is:
$$m = \frac{\Delta \ln(I_D)}{\Delta V_{GS}} = \frac{1}{\eta V_{thermal}}$$
Rearranging for $\eta$:
$$\eta = \frac{1}{m \cdot V_{thermal}}$$ \\
$$\Delta V_{GS} = 270.09751 \text{ mV} - 200.86674 \text{ mV} = 69.23077 \text{ mV} = 0.06923 \text{ V}$$
$$\Delta \ln(I_D) = -18.025813 - (-20.031662) = 2.005849$$
Slope ($m$):
$$m = \frac{2.005849}{0.06923} \approx 28.973 \text{ V}^{-1}$$

Using the standard thermal voltage at $27^{\circ}\text{C}$ ($V_{thermal} \approx 25.86 \text{ mV}$):
$$\eta = \frac{1}{28.973 \times 0.02586}$$
$$\eta \approx \mathbf{1.335}$$

\begin{figure}[H]
    \centering
    \includegraphics[width=1\textwidth]{images/2_e_2.png}
    \caption{Plot of  $\ln(I_D)$ vs. $V_{GS}$ for $V_{DS} = 1.8V$}
\end{figure}

Point 1: $(V_{GS1}, \ln(I_{D1})) = (198.91658 \text{ mV}, -18.7923)$ \\
Point 2: $(V_{GS2}, \ln(I_{D2})) = (297.39978 \text{ mV}, -16.083512)$\\

$$\Delta V_{GS} = 297.39978 \text{ mV} - 198.91658 \text{ mV} = 98.4832 \text{ mV} = 0.0984832 \text{ V}$$
$$\Delta \ln(I_D) = -16.083512 - (-18.7923) = 2.708788$$
Slope ($m$):
$$m = \frac{2.708788}{0.0984832} \approx 27.505 \text{ V}^{-1}$$

Using a standard thermal voltage at $27^{\circ}\text{C}$ ($V_{thermal} \approx 25.86 \text{ mV}$):
$$\eta = \frac{1}{27.505 \times 0.02586}$$
$$\eta \approx \mathbf{1.406}$$

\begin{figure}[H]
    \centering
    \includegraphics[width=1\textwidth]{images/2_e_sub_threshold.png}
    \caption{Subthreshold Region Marked on $I_D$ vs. $V_{GS}$ Plots}
\end{figure}
\section*{Question 3}

Finding unity current gain frequency $f_T$ of the technology node.

\begin{enumerate}[label=(\alph*)]
    \item Using minimum size NMOS, find the $f_T$ (unity current gain frequency, where $|i_d/i_g| = 1$) for the 180~nm TSMC, 180~nm SCL, and 130~nm IBM technologies. You are expected to show the method, plots, and your explanation.
    \item Suggest a method to compare the intrinsic gains $(g_m r_o)$ of minimum size transistors for the three model files given to you. Show the schematic, netlist, simulation results, and necessary discussions.
\end{enumerate}

\textit{Hint: Useful commands:}  
\verb|lib 'tsl85_scl.lib' tt_18|,  
\verb|include TSMC_180nm.txt|,  
\verb|include IBM130nm.txt|,  
\verb|use nmos4 and edmin model name as n18|,  
\verb|CMOSN130, CMOSN180, ADSP/PSPD|,  
and substitute in MOS parameters.

\section*{Solution:}
\subsection*{(a) Unity Current Gain Frequency ($f_T$) Comparison}

Theoretically, $f_T \approx \frac{g_m}{2\pi(C_{gs} + C_{gd})}$. As technology scales from 180nm to 130nm, capacitances decrease significantly, and $f_T$ increases. The simulation data even shows that 130nm node shas a much higher $f_T$ than the 180nm nodes.

\begin{table}[H]
    \centering
    \caption{Comparison of Unity Current Gain Frequency ($f_T$) for Minimum Size NMOS across Technology Nodes ($V_{GS} = 1.2\text{V}$, $V_{DS} = 1.8\text{V}$)}
    \begin{tabular}{|l|c|c|}
        \hline
        \textbf{Technology Node} & \textbf{Feature Size ($L_{min}$)} & \textbf{Extracted $f_T$ (GHz)} \\ \hline
        SCL 180nm (Real PDK)                & 180 nm                            & 64.90                          \\ \hline
        TSMC 180nm (Spice model)               & 180 nm                            & 43.26                          \\ \hline
        IBM 130nm (Spice model)                & 130 nm                            & 51.54                          \\ \hline
    \end{tabular}
\end{table}

\textbf{Note:} \\
All reported frequencies were extracted at a gate-source voltage ($V_{GS}$) of $1.2\,\text{V}$ to ensure the devices are biased in the strong inversion saturation region.
\begin{figure}[H]
    \centering
    \includegraphics[width=1\textwidth]{images/3_a_1.png}
    \caption{Plot of Gain vs. $V_{GS}$ for SCL180nm technology node}
\end{figure}

\begin{figure}[H]
    \centering
    \includegraphics[width=1\textwidth]{images/3_a_2.png}
    \caption{Plot of Gain vs. Frequency for TSMC180nm technology node}
\end{figure}

\begin{figure}[H]
    \centering
    \includegraphics[width=1\textwidth]{images/3_a_3.png}
    \caption{Plot of Gain vs. Frequency for IBM130nm technology node}
\end{figure}

\textbf{SCL 180nm vs. TSMC 180nm} \\
Even though both technologies share the same $180\,\text{nm}$ feature size, the SCL $180\,\text{nm}$ ($64.9\,\text{GHz}$) significantly outperforms the TSMC $180\,\text{nm}$ ($43.26\,\text{GHz}$). This discrepancy is likely due to differences in the process parameters provided in their respective library files. SCL is a real PDK file whereas TSMC is a spice model file. 

\textbf{Scaling Trend (180nm vs. 130nm)} \\
Theoretically, the transition frequency $f_T$ is given by:
$$f_T \approx \frac{g_m}{2\pi C_{gs}} \approx \frac{\mu_n C_{ox} \frac{W}{L} (V_{GS}-V_T)}{2\pi \cdot \frac{2}{3} W L C_{ox}} \propto \frac{\mu_n (V_{GS}-V_T)}{L^2}$$
As the channel length $L$ scales down from 180nm to 130nm, $f_T$ is expected to increase significantly due to the $1/L^2$ dependence. The carriers cross the channel faster, resulting in higher operating speeds. The general trend in semiconductor scaling dictates that smaller nodes (like 130nm) inherently offer higher intrinsic speed capabilities ($f_T$) compared to larger nodes (like 180nm), provided the device is not limited by velocity saturation or excessive parasitic capacitances.

\subsection*{(b) Intrinsic Gain ($g_m r_o$) Comparison Methodology}

To compare the intrinsic gains ($g_m r_o$), I employed a simulation method using an ideal current source load.

\textbf{Methodology:}
I biased the NMOS transistor in the saturation region using an ideal DC current source connected to the drain. The value of this current ($I_{DS}$) was chosen based on the $I_D$ vs. $V_{DS}$ characteristics to ensure saturation.

Since an ideal current source has infinite small-signal impedance ($r_{oc} \to \infty$), the effective load seen at the drain is solely the output resistance of the transistor ($r_o$).

\textbf{Mathematical Justification:}
The small-signal voltage gain of a common-source amplifier is:
$$ |A_v| = g_m (r_o || r_{oc}) $$
With $r_{oc} \to \infty$, the equation simplifies to:
$$ |A_v| = g_m r_o $$
Thus, the AC gain magnitude obtained from the simulation corresponds directly to the intrinsic gain of the transistor.

To normalize the comparison, each transistor was biased in the saturation region with $V_{DS} = 1.2\text{V}$ and $V_{GS} = 1.2\text{V}$. The gains were extracted at a frequency of 1 GHz to capture the intrinsic performance.

\begin{figure}[H]
    \centering
    \includegraphics[width=1\textwidth]{images/3_b_scl_iv.png}
    \caption{$I_D$ vs. $V_{DS}$ for SCL180nm technology node at $V_{DS} = 1.2V \& V_{GS} = 1.2V$}
\end{figure}

\begin{figure}[H]
    \centering
    \includegraphics[width=1\textwidth]{images/3_b_scl_gain.png}
    \caption{Intrinsic Gain ($g_m r_o$) for SCL180nm technology node at Midband Frequency 1MHz}
\end{figure}

\begin{figure}[H]
    \centering
    \includegraphics[width=1\textwidth]{images/3_b_tsmc_iv.png}
    \caption{$I_D$ vs. $V_{DS}$ for TSMC180nm technology node at $V_{DS} = 1.2V \& V_{GS} = 1.2V$}
\end{figure}

\begin{figure}[H]
    \centering
    \includegraphics[width=1\textwidth]{images/3_b_tsmc_gain.png}
    \caption{Intrinsic Gain ($g_m r_o$) for TSMC180nm technology node at Midband Frequency 1MHz}
\end{figure}

\begin{figure}[H]
    \centering
    \includegraphics[width=1\textwidth]{images/3_b_ibm_iv.png}
    \caption{$I_D$ vs. $V_{DS}$ for IBM130nm technology node at $V_{DS} = 1.2V \& V_{GS} = 1.2V$}
\end{figure}

\begin{figure}[H]
    \centering
    \includegraphics[width=1\textwidth]{images/3_b_ibm.png}
    \caption{Intrinsic Gain ($g_m r_o$) for IBM130nm technology node at Midband Frequency 1MHz}
\end{figure}

\begin{table}[H]
\centering
\caption{Intrinsic Gain Comparison Across Technology Nodes}
\resizebox{\textwidth}{!}{%
\begin{tabular}{|l|c|c|c|c|c|}
\hline
\textbf{Technology Node} & \textbf{$L_{\min}$} & \textbf{$W$ used} & \textbf{Bias Current ($I_D$)} & \textbf{Intrinsic Gain (dB)} & \textbf{Intrinsic Gain (V/V)} \\
\hline
SCL 180\,nm  & 180\,nm & 0.45\,$\mu$m & 116.44\,$\mu$A & 25.99 & $\approx 19.93$ \\
TSMC 180\,nm & 180\,nm & 0.45\,$\mu$m & 147.75\,$\mu$A & 18.77 & $\approx 8.68$ \\
IBM 130\,nm  & 130\,nm & 0.325\,$\mu$m & 209.93\,$\mu$A & -1.67 & $\approx 0.82$ \\
\hline
\end{tabular}%
}
\label{tab:intrinsic_gain_comparison}
\end{table}

\section*{Question 4}
Using TSMC model file, draw a circuit for an analog amplifier using complementary NMOS ($5\,\mu\text{m}/1\,\mu\text{m}$) and PMOS ($10\,\mu\text{m}/1\,\mu\text{m}$) configuration discussed in class. Use only one DC source $V_{DD} = 1.8~\text{V}$ and generate other voltages using resistive dividers.

\noindent
From simulations, find the DC bias point and maximum input amplitude ($A_{\text{max}}$) for maximum gain and THD $< 10\%$. Report the bias point and the maximum gain.
\noindent
How will you cascade two such amplifiers? Show the schematic with the exact schematic and simulation results (gain and THD) for the cascaded amplifier at $A_{\text{max}}$.

\section*{Solution:}
The amplifier was designed using a complementary NMOS ($5\mu\text{m}/1\mu\text{m}$) and PMOS ($10\mu\text{m}/1\mu\text{m}$) configuration under a $1.8\text{V}$ supply. To determine the optimal bias point, a DC sweep of $V_{GS}$ was performed, and the derivative of the transfer characteristic ($dV_{out}/dV_{in}$) was analyzed to identify the region of maximum slope. The peak gain was found at an input bias of approximately $811.65\text{mV}$. This was implemented using a resistive divider with $R_3 = 16.2\text{k}\Omega$ and $R_4 = 13.3\text{k}\Omega$

\begin{figure}[H]
    \centering
    \includegraphics[width=1\textwidth]{images/4_bias.png}
    \caption{Biasing Circuit for Single Stage CMOS Amplifier in TSMC 180nm Technology Node}
\end{figure}
 The single-stage configuration achieved a maximum small-signal gain of $20.54\text{dB}$ ($\approx 10.64\text{V/V}$) at $1\text{kHz}$.
\begin{figure}[H]
    \centering
    \includegraphics[width=1\textwidth]{images/4_max_gain.png}
    \caption{AC Small Signal Gain for Single Stage CMOS Amplifier in TSMC 180nm Technology Node}
\end{figure}
Transient analysis showed that for a $10\text{mV}$ input sine wave, the Total Harmonic Distortion (THD) remained low at $0.37\%$. The maximum input amplitude ($A_{max}$) for a single stage was identified as $90\text{mV}$, yielding a THD of $9.97\%$, effectively meeting the $<10\%$ requirement.
\begin{figure}[H]
    \centering
    \includegraphics[width=1\textwidth]{images/4_matched_max_gain.png}
    \caption{Maximum Input Amplitude for Maximum Gain Point for Single Stage CMOS Amplifier in TSMC 180nm Technology Node}
\end{figure}

\begin{figure}[H]
    \centering
    \includegraphics[width=1\textwidth]{images/4_max_THD.png}
    \caption{Maximum Input Amplitude for THD $< 10\%$ in Single Stage CMOS Amplifier in TSMC 180nm Technology Node}
\end{figure}
Two identical stages were cascaded using a $1\mu\text{F}$ AC coupling capacitor to ensure DC isolation between stages while allowing the signal to pass. To mitigate the loading effect—where the second stage's input impedance reduces the gain of the first stage—the resistive divider values were scaled by a factor of 10 to $162\text{k}\Omega$ and $133\text{k}\Omega$. This increased the input impedance of the second stage to approximately $73\text{k}\Omega$, preserving the overall gain.
\begin{figure}[H]
    \centering
    \includegraphics[width=1\textwidth]{images/4_cascode.png}
    \caption{Maximum input amplitude for Maximum gain for Cascaded Two Stage CMOS Amplifier Schematic in TSMC 180nm Technology Node}
\end{figure}

\begin{figure}[H]
    \centering
    \includegraphics[width=1\textwidth]{images/4_cascode_bode.png}
    \caption{AC Small Signal Gain for Cascaded Two Stage CMOS Amplifier in TSMC 180nm Technology Node}    
\end{figure}

\begin{figure}[H]
    \centering
    \includegraphics[width=1\textwidth]{images/4_cascode_max_THD.png}
    \caption{Maximum Input Amplitude for THD $< 10\%$ in Cascaded Two Stage CMOS Amplifier in TSMC 180nm Technology Node}
\end{figure}
The cascaded system provided a significantly higher gain of $39.56\text{dB}$ ($\approx 95\text{V/V}$). However, it was observed that the $90\text{mV}$ input signal ($A_{max}$ of a single stage) caused severe output clipping in the cascaded setup, resulting in a THD of $36.76\%$. Consequently, for the cascaded amplifier to operate within the $10\%$ THD limit, the input amplitude must be scaled down to accommodate the increased gain.

\begin{table}[H]
\centering
\caption{Performance Comparison at Different Input Amplitudes}
\label{tab:input_comparison}
\begin{tabular}{|l|c|c|}
\hline
\textbf{Parameter} & \textbf{Value at 10\,mV Input} & \textbf{Value at 90\,mV Input} \\
\hline
Output Swing & Unclipped (approx.\ 1.32\,V\textsubscript{pp}) & Severely Clipped (1.8\,V flat peaks) \\
\hline
Total Harmonic Distortion & 9.68\% (Success) & 36.76\% (Fail) \\
\hline
Cascaded Gain & 39.56\,dB & N/A (Non-linear due to clipping) \\
\hline
\end{tabular}
\end{table}

\end{document}