\documentclass[12pt,a4paper]{article}
\usepackage[utf8]{inputenc}
\usepackage{amsmath, amssymb}
\usepackage{enumitem}
\usepackage{geometry}
\geometry{margin=1in}

\title{Assignment 1 Solutions}
\author{Chamarthy Madhan Sai Krishna - 2023102030}
\date{\today}

\begin{document}

\maketitle

\section*{Question 2}
\subsection*{a}
\subsection*{b}
\subsection*{c}
The Observation:
You will likely observe that the threshold voltage ($V_T$) estimated for Case (b) ($V_{DS} = 1.8V$) is lower than the $V_T$ estimated for Case (a) ($V_{DS} = 50mV$).
\textbf{Drain-Induced Barrier Lowering (DIBL)}In a modern short-channel MOSFET, as the drain voltage ($V_{DS}$) increases, the depletion region around the drain extends further into the channel toward the source. This drain depletion region helps the gate "pull down" the potential barrier between the source and the channel. Because the drain is effectively helping the gate turn the transistor on, a lower gate voltage ($V_{GS}$) is required to create the inversion layer, resulting in a lower apparent $V_T$.
\subsection*{d}
\subsection*{e}
Estimating the Subthreshold Slope Factor ($\eta$):Using the provided natural exponential relationship $I_D = I_0 \exp\left( \frac{V_{GS}}{\eta V_{thermal}} \right)$, we can derive $\eta$ using the slope of your graph:Convert to Natural Log: $\ln(I_D) = \ln(I_0) + \frac{V_{GS}}{\eta V_{thermal}}$Find the Slope ($m$): Pick two points on the straight-line portion of your subthreshold curve: $(V_{GS1}, \ln(I_{D1}))$ and $(V_{GS2}, \ln(I_{D2}))$.$$m = \frac{\ln(I_{D2}) - \ln(I_{D1})}{V_{GS2} - V_{GS1}}$$Solve for $\eta$: Since the slope $m = \frac{1}{\eta V_{thermal}}$:$$\eta = \frac{1}{m \cdot V_{thermal}}$$(Note: Use $V_{thermal} \approx 25.9mV$ at room temperature).
\section*{Question 3}
\subsection*{a}
Theoretically, $f_T \approx \frac{g_m}{2\pi(C_{gs} + C_{gd})}$. As technology scales from 180nm to 130nm, capacitances decrease significantly, and $f_T$ increases. You should expect the 130nm node to have a much higher $f_T$ than the 180nm nodes.
\subsection*{b}
Scaling vs. Gain: You will likely observe that the 130nm node has the lowest intrinsic gain. This is because as $L$ decreases, the Channel Length Modulation ($\lambda$) effect becomes more severe, increasing $g_{ds}$ (lowering $r_o$) and killing the gain.TSMC vs. SCL: Even though both are 180nm, they will have different doping profiles and oxide thicknesses. The one with the lower $g_{ds}$ (higher $r_o$) is "better" for high-gain analog design.Hand Calculation Model: In simple models, we use $A_v = \frac{2}{\lambda (V_{GS}-V_T)}$. This suggests that to maximize gain in any node, you should use lower overdrive voltages and longer channel lengths.

\section*{Question 4}


Cascading Two StagesWhen cascading, you cannot connect the output of Stage 1 directly to the gate of Stage 2 because their DC levels might conflict. You must use AC Coupling.Cascaded Schematic:Stage 1: Identical to the first circuit.Interstage Coupling: Connect a capacitor ($C_{cascade}$) from the output of Stage 1 to the gate of Stage 2.Stage 2 Biasing: Stage 2 needs its own resistive divider to set its gate to $0.9V$.Expected Cascaded Results:Total Gain: $A_{total} = A_{v1} \times A_{v2}$. If one stage gives $-30$, the cascade should give roughly $+900$ V/V (Note the phase becomes positive).THD for $A_{max}$: Because the gain is now much higher, the $A_{max}$ you found for a single stage will cause massive clipping in the second stage.Observation: You will find that for the two-stage amplifier, the new $A_{max}$ is significantly smaller (often in the micro-volt range) to keep the second stage from saturating and keeping THD < 10%.

\end{document}